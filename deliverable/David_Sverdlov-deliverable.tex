\documentclass[12pt,portrait]{article}
\usepackage{geometry}

\usepackage{tikz}
\usepackage{url}
\usepackage{hyperref}
%\geometry{verbose,letterpaper}


\newcommand{\imgurl}{../images/} %%%%%%%%%%%%%%%%%%%%%%%%%%%%%%%%%%%%%%%%%%%%%%%%%%%%%%%%%%%%%%%%%%%%%%%%
\newcommand{\weburl}[1]{\url{https://raw.githubusercontent.com/dsverdlo/AMuRate_CSSE/master/images/#1}}


\author{David Sverdlov, 94351, dsverdlo@vub.ac.be}
\title{Capita Selecta Software Engineering \\
 Project 2015} %%%%%%%%%%%%%%%%%%%%%%%%%%%%%%%%%%%%%%%%%%%%%%%%%%%%%%%%%%%%%%%%%%%%%%%%%%%%%%%%%%%

\begin{document}
%\includegraphics[scale=0.5]{VUB}
\maketitle

\begin{tikzpicture}[remember picture, overlay]
 \node [anchor=north west, inner sep=30pt]  at (current page.north west)
     {\includegraphics[height=2.2cm]{VUB}};
\end{tikzpicture}

\section{Introduction}

This deliverable documents the different steps taken in the project for the course Capita Selecta of Software Engineering, taught by Maja D'Hondt at the Vrije Universiteit Brussel. The goal of the project is to improve an existing program by means of automated unit testing and refactoring. The project is individual unless a team of 2 people has been approved. The chosen project is the Android application 'AMuRate', which was a part of my bachelor's project. This has been proposed and approved in one of the first lectures and will be first be described before moving on to the different steps of this project.

\section{AMuRate}

AMuRate is the name of the Android application, developed to collect music ratings and ultimately present new music according to the taste of the users. The app is developed for Android 4 and provides an easy to use user interface. Users do not even have to sign up to be able to use the app because there are identified by their phone id. The first thing users have to do is enter some search terms. These can be partial or full keywords, and there is an input field for artist names, and one for song names. Users can choose if they fill in one or two fields (if they do not enter any information a pop up will notify them that they might have forgotten this step). The app the queries the Last.FM database and will display a list of the results. Last.FM has been chosen because they provide a free to use API and has an enormous music database. Users can click on any of the results and view the details of a chosen artist or song. The average rating is shown and the user can give a rating of their own, on a scale of 0 - 5. These ratings are stored on my database, with the purpose of applying recommendation algorithms on this data. This way we could recommend new music to our users according to their taste. The implementation and employment of the recommendation systems fell out of the scope of the bachelor project. For more information about the application I refer to the project report\footnote{\url{https://github.com/dsverdlo/AMuRate/blob/master/Log/LaTeX/Bachelorproef.pdf}}.

\section{Project Setup}

The setup for this project will consist of a source code repository on Github, located at \url{https://github.com/dsverdlo/AMuRate_CSSE}. This report will be written in LaTeX and will also be available on the repository. The story maps will be created on StoriesOnBoard\footnote{\url{http://storiesonboard.com/}} and occassionaly uploaded to track progress. 

\section{Story Map}

As mentioned before, the story map will be created with the tool StoriesOnBoard. This tool provides the means to compose a big map of all required activities and tasks. The map can be exported as a XML, JPG or PDF file, and releases can be set. For the first step in this project, we have mapped our stories to this tool. The result has been exported and can be seen below.

\begin{figure}[h]
\includegraphics[scale=1.2]{\imgurl storymap1}
\caption{\weburl{storymap1.jpg} }
\end{figure}
 
 
\section{Identifying risks}

This part was not that simple since the tool does not allow to mark certain tasks. To do this we could either set a task to 'done' to give it a small red label, or we could simply edit the JPG file in paint. We have chosen for the latter since we can add star shapes and make the risks more visible. In the future we might opt to use a more excessive photo manipulating tool, like GIMP, to support layers and be able to move stars, reshape them and keep them equally sized.

The risks can be seen in the following figure:
\begin{figure}[h]
	\includegraphics[scale=1.2]{\imgurl storymap1risks}
	\caption{\weburl{storymap1risks.jpg}}
\end{figure}

%\section{Slicing}
\section{Test Strategy}


\section{Planning}

Since I am currently working on a thesis and 2 other projects, I will not be able to spend more time than the 6 ECTS worth. The schedules for the other courses are planned only a couple of weeks in  advance so it is rather difficult to determine a steady schedule at this moment.


\section{Detailed planning for next iteration}

The first item on the planning is to slice the story map up and set deadlines for their releases. Then to write about the test strategy and time allocation for each part and execute said strategy.









%\newpage
%\section{Screenshots}
%\animategraphics[width=\linewidth]{0.5}{\taskimgname}{1}{10}
\end{document}